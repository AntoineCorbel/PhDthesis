\chapter*{Introduction}
\addcontentsline{toc}{chapter}{Introduction}

Gamma-ray imaging consists in reconstructing the position of a (radioactive) source emitting gamma-rays
at energies ranging from a few keV to a few MeV. This process is used in a wide variety of applications such 
as astrophysics (space observations), nuclear safety \cite{Gagliardi2024NovelAO}, homeland security and medicine through positron-emission tomography (PET)
and single photon emission computed tomography (SPECT).

Most often gamma-ray imaging devices are composed of two layers of a material, often called scatterer and absorber. Most used 
materials are NaI, CZT (Cadmium Zinc Tellurium), having different strengths:
\begin{itemize}
    \item NaI\@: low cost and decent energy resolution (6-7\% at 662 keV);
    \item CZT\@: room temperature semi-conductor and excellent energy resolution (1-2\% at 662 keV);
\end{itemize}

This work studies the imaging capabilities of a single volume of segmented hyper pure germanium (HPGe). HPGe has
the best energy resolution of all the materials presented above with ~0.15\% at 662 keV. Being a semi-conductor with a small
band gap energy, it has to be cooled to avoid current leakage due to thermal excitation.

\[ x = \frac{-b \pm \sqrt{b^2 - 4ac}}{2a} \]
